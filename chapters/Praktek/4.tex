\section{D. Irga B. Naufal Fakhri}
\subsection{Soal 1}
Buatlah fungsi untuk membuka file csv dengan lib csv mode list
\lstinputlisting[firstline=9, lastline=20]{src/4/1174066/Praktek/lib_1174066_csv.py}

\subsection{Soal 2}
Buatlah fungsi untuk membuka file csv dengan lib csv mode dictionary
\lstinputlisting[firstline=24, lastline=34]{src/4/1174066/Praktek/lib_1174066_csv.py}

\subsection{Soal 3}
Buatlah fungsi untuk membuka file csv dengan lib pandas mode list
\lstinputlisting[firstline=9, lastline=11]{src/4/1174066/Praktek/lib_1174066_pandas.py}

\subsection{Soal 4}
Buatlah fungsi untuk membuka file csv dengan lib pandas mode dictionary
\lstinputlisting[firstline=15, lastline=19]{src/4/1174066/Praktek/lib_1174066_pandas.py}

\subsection{Soal 5}
Buat fungsi baru untuk mengubah format tanggal menjadi standar dataframe
\lstinputlisting[firstline=22, lastline=24]{src/4/1174066/Praktek/lib_1174066_pandas.py}

\subsection{Soal 6}
Buat fungsi baru  untuk mengubah index kolom
\lstinputlisting[firstline=28, lastline=30]{src/4/1174066/Praktek/lib_1174066_pandas.py}

\subsection{Soal 7}
Buat fungsi baru untuk mengubah atribut atau nama kolom
\lstinputlisting[firstline=35, lastline=37]{src/4/1174066/Praktek/lib_1174066_pandas.py}

\subsection{Soal 8}
Buat program main yang menggunakan library NPM csv yang membuat dan membaca file csv
\lstinputlisting[caption=lib\textunderscore 1174066\textunderscore csv.py, firstline=38, lastline=43]{src/4/1174066/Praktek/lib_1174066_csv.py}

\lstinputlisting[caption=main.py, firstline=8, lastline=17]{src/4/1174066/Praktek/main.py}

\subsection{Soal 9}
Buat program main2.py yang menggunakan library NPM pandas.py yang membuat dan membaca file csv
\lstinputlisting[caption=main2.py, firstline=42, lastline=48]{src/4/1174066/Praktek/lib_1174066_pandas.py}
\lstinputlisting[caption=main2.py, firstline=8, lastline=17]{src/4/1174066/Praktek/main2.py}

\subsection{Keterampilan Penanganan Error}
Pada praktikum saat ini saya tidak mendapatkan error

%%%%%%%%%%%%%%%%%%%%%%%%%%%%%%%%%%%%%%%%%%%%%%%%%%%%%%%%%%%%%%%%%%%%%%%%%%%%%%%%%%%%%%%%%%%%%%%%%%%%%%%%%%%%%%%%%%%%%%%%%%
\section{Fanny Shafira Damayanti | 1174069}
\subsection{Keterampilan Pemrograman}
\begin{enumerate}
	\item NO 1 

	\lstinputlisting[firstline=10, lastline=15]{src/4/1174069/Praktek/1174069_csv.py}

	\item NO 2

	\lstinputlisting[firstline=17, lastline=22]{src/4/1174069/Praktek/1174069_csv.py}

	\item NO 3 

	\lstinputlisting[firstline=10, lastline=13]{src/4/1174069/Praktek/1174069_pandas.py}

	\item NO 4 

	\lstinputlisting[firstline=10, lastline=13]{src/4/1174069/Praktek/1174069_pandas.py}

	\item NO 5 

	\lstinputlisting[firstline=15, lastline=19]{src/4/1174069/Praktek/1174069_pandas.py}

	\item NO 6 

	\lstinputlisting[firstline=21, lastline=24]{src/4/1174069/Praktek/1174069_pandas.py}

	\item NO 7 

	\lstinputlisting[firstline=26, lastline=30]{src/4/1174069/Praktek/1174069_pandas.py}

	\item NO 8

	\lstinputlisting[firstline=8, lastline=13]{src/4/1174069/Praktek/main.py}

	\item NO 9 

	\lstinputlisting[firstline=8, lastline=13]{src/4/1174069/Praktek/main2.py}

\end{enumerate}

\subsection{Penanganan Error}
\begin{enumerate}
	\item Peringatan error yang terdapat pada praktikum chapter 4 ini yaitu :

	\begin{itemize}
		\item Syntax Errors
		Syntax Errors terjadi ketika ada kesalahan dalam meuliskan kode. Solusinya adalah memperbaiki penulisan kode yang salah.

		\item Name Error
		NameError terjadi ketika salah mengetikan kode local name yang tidak terdefinisi. Solusinya adalah menuliskan kode dengan benar agar function nya dapat terpanggil. 

		\item Type Error
		TypeError terjadi pada saat eksekusi terhadapt fungsi dengan tipe objek tidak sesuai. Solusinya mengkonversi variablenya harus sesuai dengan tipe datanya.
	\end{itemize}

	Contoh Penggunaan TryExcept
	\lstinputlisting[firstline=55, lastline=67]{src/4/1174069/Praktek/1174069.py}
\end{enumerate}
